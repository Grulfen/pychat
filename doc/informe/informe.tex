\documentclass[a4paper, 12pt]{article}
\usepackage[utf8]{inputenc}
\usepackage[spanish]{babel}
\usepackage{fancyhdr}
\usepackage{gantt}
\pagestyle{fancy}

\title{Informe - pychat}
\author{Gustav Svensk}
\begin{document}
\lhead{Gustav Svensk}
\rhead{pychat}
\cfoot{\thepage}
\renewcommand{\headrulewidth}{0.4pt}
% \renewcommand{\footrulewidth}{0.4pt}


\maketitle
\thispagestyle{empty}
\begin{center}
        {\large Versión 0.1}
\end{center}
\newpage

\setcounter{page}{1}
\tableofcontents
\listoftables
\listoffigures
\newpage

\section{Resumen}
\subsection{Equipo}
Sólo hay un integrante del equipo, Gustav Svensk, que tendrá todos los roles del
equipo.
\subsection{Descripción}
El proyecto va a consistir en un programa chat de tipo P2P. Sólo se
podrá chatear entre dos personas (no puede estar 3 personas en el mismo
chat).

El programa tendrá una interfaz básica basado en parte en la linea de comandos.

El programa será escrito en Python.
\subsection{Requerimientos}
En esta sección se escribe los requerimientos del programa.
\subsubsection{HW}
\begin{itemize}
        \item Computadora conectado al Internet capaz a correr python
        \item Puede ser que se necesitará acceso a un router para cambiar el
                reenvío de puertos.
\end{itemize}
\subsubsection{SW}
El programa final va a incluir un archivo setup.py para facilitar la
instalación.
\begin{itemize}
        \item Python 3.x
        \item PyQt4
        \item Qt4
\end{itemize}
\subsection{Tablas GANTT}
\subsubsection{Tabla inicial}
\begin{gantt}[xunitlength=5mm]{8}{15}
\begin{ganttitle}
        \numtitle{2}{2}{30}{1}
\end{ganttitle}
\ganttbar{Recibir mensaje}{0}{3}
\ganttbarcon{Enviar mensaje}{3}{2}
\ganttbarcon{Parse mensaje}{5}{1}
\ganttbar{GUI}{6}{2}
\ganttbar{Consola}{8}{2}
\ganttbar{Concurrencia}{10}{3}
\ganttbarcon{Verificación}{13}{2}
\ganttcon{5}{2}{13}{7}
\end{gantt}

\subsubsection{Tabla final}
\begin{gantt}[xunitlength=5mm]{8}{19}
\begin{ganttitle}
        \numtitle{2}{2}{38}{1}
\end{ganttitle}
\ganttbar{Recibir mensaje}{0}{4}
\ganttbarcon{Enviar mensaje}{4}{2}
\ganttbarcon{Parse mensaje}{6}{1}
\ganttbar{GUI}{7}{2}
\ganttbar{Consola}{9}{2}
\ganttbar{Concurrencia}{11}{4}
\ganttbarcon{Verificación}{15}{4}
\ganttcon{5}{2}{15}{7}
\end{gantt}
\subsection{Resultados}

\section{Arquitectura}
\section{Diseño de Protocolo}
\section{Diseño de Código}
\section{Verificación}

\end{document}

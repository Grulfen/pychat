\documentclass[a4paper, 12pt]{article}
\usepackage[utf8]{inputenc}
\usepackage[spanish]{babel}
\usepackage{fancyhdr}
\pagestyle{fancy}

\title{Propuesta - pychat}
\author{Gustav Svensk}
\begin{document}
\lhead{Gustav Svensk}
\rhead{pychat}
\cfoot{\thepage}
\renewcommand{\headrulewidth}{0.4pt}
% \renewcommand{\footrulewidth}{0.4pt}


\maketitle
\thispagestyle{empty}
\begin{center}
        {\large Versión 0.1}
\end{center}
\newpage
\setcounter{page}{1}

\section{Equipo}
Sólo hay un integrante del equipo, Gustav Svensk, que tendrá todos los roles del
equipo.
\section{Descripción}
El proyecto va a consistir en un programa chat de tipo P2P. Sólo se
podrá chatear entre dos personas (no puede estar 3 personas en el mismo
chat).

El programa tendrá una interfaz básica basado en parte en la linea de comandos.

El programa será escrito en Python.
\section{Requerimientos}
En esta sección se escribe los requerimientos del programa.
\subsection{HW}
\begin{itemize}
        \item Computadora conectado al Internet capaz a correr python
        \item Puede ser que se necesitará acceso a un router para cambiar el
                reenvío de puertos.
\end{itemize}
\subsection{SW}
El programa final va a incluir un archivo setup.py para facilitar la
instalación.
\begin{itemize}
        \item Python 3.x
        \item PyQt4
        \item Qt4
\end{itemize}
\section{Tabla GANTT}
%TODO fråga om det här verkligen är nödvändigt

\end{document}
